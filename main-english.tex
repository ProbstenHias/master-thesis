% !TeX spellcheck = en-US
% !TeX encoding = utf8
% !TeX program = pdflatex
% !BIB program = biber
% -*- coding:utf-8 mod:LaTeX -*-
% !TEX root = ./main-english.tex

% vv  scroll down to line 200 for content  vv


\let\ifdeutsch\iffalse
\let\ifenglisch\iftrue
% EN: This file is loaded before the \documentclass command in the main document

% EN: The following package allows \\ at the title page
%     For more information see https://github.com/latextemplates/scientific-thesis-cover/issues/4
\RequirePackage{kvoptions-patch}

\ifenglisch
  \PassOptionsToClass{numbers=noenddot}{scrbook}
\else
  %()Aus scrguide.pdf - der Dokumentation von KOMA-Script)
  %Nach DUDEN steht in Gliederungen, in denen ausschließlich arabische Ziffern für die Nummerierung
  %verwendet werden, am Ende der Gliederungsnummern kein abschließender Punkt
  %(siehe [DUD96, R3]). Wird hingegen innerhalb der Gliederung auch mit römischen Zahlen
  %oder Groß- oder Kleinbuchstaben gearbeitet, so steht am Ende aller Gliederungsnummern ein
  %abschließender Punkt (siehe [DUD96, R4])
  \PassOptionsToClass{numbers=autoendperiod}{scrbook}
\fi

% Warns about outdated packages and missing caption declarations
% See https://www.ctan.org/pkg/nag
\RequirePackage[l2tabu, orthodox]{nag}

%DE: Neue deutsche Trennmuster
%    Siehe http://www.ctan.org/pkg/dehyph-exptl und http://projekte.dante.de/Trennmuster/WebHome
%    Nur für pdflatex, nicht für lualatex
\RequirePackage{ifluatex}
\ifluatex
  % do not load anything
\else
  \ifdeutsch
    \RequirePackage[ngerman=ngerman-x-latest]{hyphsubst}
  \fi
\fi

\documentclass[
  % fontsize=11pt is the standard
  a4paper,  % Standard format - only KOMAScript uses paper=a4 - https://tex.stackexchange.com/a/61044/9075
  twoside,  % we are optimizing for both screen and two-side printing. So the page numbers will jump, but the content is configured to stay in the middle (by using the geometry package)
  bibliography=totoc,
  %               idxtotoc,   %Index ins Inhaltsverzeichnis
  %               liststotoc, %List of X ins Inhaltsverzeichnis, mit liststotocnumbered werden die Abbildungsverzeichnisse nummeriert
  headsepline,
  cleardoublepage=empty,
  parskip=half,
  %               draft    % um zu sehen, wo noch nachgebessert werden muss - wichtig, da Bindungskorrektur mit drin
  draft=false
]{scrbook}
% !TeX encoding = utf8
% -*- coding:utf-8 mod:LaTeX -*-

% EN: This file includes basic packages and sets options. The order of package
%     loading is important

% DE: In dieser Datei werden zuerst die benoetigten Pakete eingebunden und
%     danach diverse Optionen gesetzt. Achtung Reihenfolge ist entscheidend!


% EN: Styleguide:
% - English comments are prefixed with "EN", German comments are prefixed with "DE"
% - Prefixed headings define the language for the subsequent paragraphs
% - It is tried to organize packages in blocks. Bocks are separated by two empty lines.

% DE: Styleguide:
%
% Ein sehr kleiner Styleguide. Packages werden in Blöcken organisiert.
% Zwischen zwei Blöcken sind 2 Leerzeilen!


% EN: Enable copy and paste of text from the PDF
%     Only required for pdflatex. It "just works" in the case of lualatex.
%     mmap enables mathematical symbols, but does not work with the newtx font set
%     See: https://tex.stackexchange.com/a/64457/9075
%     Other solutions outlined at http://goemonx.blogspot.de/2012/01/pdflatex-ligaturen-und-copynpaste.html and http://tex.stackexchange.com/questions/4397/make-ligatures-in-linux-libertine-copyable-and-searchable
%     Trouble shooting outlined at https://tex.stackexchange.com/a/100618/9075

\ifluatex
\else
  \usepackage{cmap}
\fi


% EN: File encoding
% DE: Codierung
%     Wir sind im 21 Jahrhundert, utf-8 löst so viele Probleme.
%
% Mit UTF-8 funktionieren folgende Pakete nicht mehr. Bitte beachten!
%   * fancyvrb mit §
%   * easylist -> http://www.ctan.org/tex-archive/macros/latex/contrib/easylist/
\ifluatex
  % EN: See https://tex.stackexchange.com/a/158517/9075
  %     Not required, because of usage of fontspec package
  %\usepackage[utf8]{luainputenc}
\else
  \usepackage[utf8]{inputenc}
\fi


% DE: Parallelbetrieb tex4ht und pdflatex

\makeatletter
\@ifpackageloaded{tex4ht}{
  \def\iftex4ht{\iftrue}
}{
  \def\iftex4ht{\iffalse}
}
\makeatother


% EN: Mathematics
% DE: Mathematik
%
% DE: Viele Mathematik-Sachen. Siehe https://texdoc.net/pkg/amsmath
%
% EN: Options must be passed this way, otherwise it does not work with glossaries
% DE: fleqn (=Gleichungen linksbündig platzieren) funktioniert nicht direkt. Es muss noch ein Patch gemacht werden:
\PassOptionsToPackage{fleqn,leqno}{amsmath}
%
% DE: amsmath Muss nicht mehr geladen werden, da es von newtxmath automatisch geladen wird
% \usepackage{amsmath}


%% EN: Fonts
%% DE: Schriften
%%
%% !!! If you change the font, be sure that words such as "workflow" can
%% !!! still be copied from the PDF. If this is not the case, you have
%% !!! to use glyphtounicode. See comment at cmap package


% EN: Times Roman for all text
\ifluatex
  \RequirePackage{amsmath}
  \RequirePackage{unicode-math}
  \setmainfont{TeX Gyre Termes}
  \setmathfont{texgyretermes-math.otf}
  \setsansfont[Scale=.9]{TeX Gyre Heros}
  \setmonofont[StylisticSet={1,3},Scale=.9]{inconsolata}
\else
  \RequirePackage{newtxtext}
  \RequirePackage{newtxmath}
  % EN: looks good with times, but no equivalent for lualatex found,
  %     therefore replaced with inconsolata
  %\RequirePackage[zerostyle=b,scaled=.9]{newtxtt}
  \RequirePackage[varl,scaled=.9]{inconsolata}

  % DE: Symbole
  % unicode-math scheint für die meisten schon etwas anzubieten
  %
  %\usepackage[geometry]{ifsym} % \BigSquare

  % EN: The euro sign
  % DE: Das Euro Zeichen
  %     Fuer Palatino (mathpazo.sty): richtiges Euro-Zeichen
  %     Alternative: \usepackage{eurosym}
  \newcommand{\EUR}{\ppleuro}
\fi


% DE: Noch mehr Symbole
%\usepackage{stmaryrd} %fuer \ovee, \owedge, \otimes
%\usepackage{marvosym} %fuer \Writinghand %patched to not redefine \Rightarrow
%\usepackage{mathrsfs} %mittels \mathscr{} schoenen geschwungenen Buchstaben erzeugen
%\usepackage{calrsfs} %\mathcal{} ein bisserl dickeren buchstaben erzeugen - sieht net so gut aus.

% EN: Fallback font - if the subsequent font packages do not define a font (e.g., monospaced)
%     This is the modern package for "Computer Modern".
%     In case this gets activated, one has to switch from cmap package to glyphtounicode (in the case of pdflatex)
% DE: Fallback-Schriftart
%\usepackage[%
%    rm={oldstyle=false,proportional=true},%
%    sf={oldstyle=false,proportional=true},%
%    tt={oldstyle=false,proportional=true,variable=true},%
%    qt=false%
%]{cfr-lm}

% EN: Headings are typset in Helvetica (which is similar to Arial)
% DE: Schriftart fuer die Ueberschriften - ueberschreibt lmodern
%\usepackage[scaled=.95]{helvet}

% DE: Für Schreibschrift würde tun, muss aber nicht
%\usepackage{mathrsfs} %  \mathscr{ABC}

% EN: Font for the main text
% DE: Schriftart fuer den Fliesstext - ueberschreibt lmodern
%     Linux Libertine, siehe http://www.linuxlibertine.org/
%     Packageparamter [osf] = Minuskel-Ziffern
%     rm = libertine im Brottext, Linux Biolinum NICHT als serifenlose Schrift, sondern helvet (von oben) beibehalten
%\usepackage[rm]{libertine}

% EN: Alternative Font: Palantino. It is recommeded by Prof. Ludewig for German texts
% DE: Alternative Schriftart: Palantino, Packageparamter [osf] = Minuskel-Ziffern
%     Bitte nur in deutschen Texten
%\usepackage{mathpazo} %ftp://ftp.dante.de/tex-archive/fonts/mathpazo/ - Tipp aus DE-TEX-FAQ 8.2.1

% DE: Schriftart fuer Programmcode - ueberschreibt lmodern
%     Falls auskommentiert, wird die Standardschriftart lmodern genommen
%     Fuer schreibmaschinenartige Schluesselwoerter in den Listings - geht bei alten Installationen nicht, da einige Fontshapes (<>=) fehlen
%\usepackage[scaled=.92]{luximono}
%\usepackage{courier}
% DE: BeraMono als Typewriter-Schrift, Tipp von http://tex.stackexchange.com/a/71346/9075
%\usepackage[scaled=0.83]{beramono}

% EN: backticks (`) are rendered as such in verbatim environments.
%     See following links for details:
%     - https://tex.stackexchange.com/a/341057/9075
%     - https://tex.stackexchange.com/a/47451/9075
%     - https://tex.stackexchange.com/a/166791/9075
\usepackage{upquote}

% EN: For \texttrademark{}
\usepackage{textcomp}

% EN: name-clashes von marvosym und mathabx vermeiden:
\def\delsym#1{%
  %  \expandafter\let\expandafter\origsym\expandafter=\csname#1\endcsname
  %  \expandafter\let\csname orig#1\endcsname=\origsym
  \expandafter\let\csname#1\endcsname=\relax
}

%\usepackage{pifont}
%\usepackage{bbding}
%\delsym{Asterisk}
%\delsym{Sun}\delsym{Mercury}\delsym{Venus}\delsym{Earth}\delsym{Mars}
%\delsym{Jupiter}\delsym{Saturn}\delsym{Uranus}\delsym{Neptune}
%\delsym{Pluto}\delsym{Aries}\delsym{Taurus}\delsym{Gemini}
%\delsym{Rightarrow}
%\usepackage{mathabx} - Ueberschreibt leider zu viel - und die \le-Zeichen usw. sehen nicht gut aus!


% EN: Modern font encoding
%     Has to be loaded AFTER any font packages. See https://tex.stackexchange.com/a/2869/9075.
\ifluatex
\else
  \usepackage[T1]{fontenc}
\fi
%


% EN: Character protrusion and font expansion. See http://www.ctan.org/tex-archive/macros/latex/contrib/microtype/
% DE: Optischer Randausgleich und Grauwertkorrektur

\usepackage[
  babel=true, % EN: Enable language-specific kerning. Take language-settings from the languge of the current document (see Section 6 of microtype.pdf)
  expansion=alltext,
  protrusion=alltext-nott, % EN: Ensure that at listings, there is no change at the margin of the listing
  final % EN: Always enable microtype, even if in draft mode. This helps finding bad boxes quickly.
        %     In the standard configuration, this template is always in the final mode, so this option only makes a difference if "pros" use the draft mode
]{microtype}


% EN: \texttt{test -- test} keeps the "--" as "--" (and does not convert it to an en dash)
\DisableLigatures{encoding = T1, family = tt* }

% DE: fuer microtype
% DE: tracking=true muss als Parameter des microtype-packages mitgegeben werden
% DE: Deaktiviert, da dies bei Algorithmen seltsam aussieht

%\DeclareMicrotypeSet*[tracking]{my}{ font = */*/*/sc/* }%
%\SetTracking{ encoding = *, shape = sc }{ 45 }
% DE: Hier wird festgelegt,
%     dass alle Passagen in Kapitälchen automatisch leicht
%     gesperrt werden.
%     Quelle: http://homepage.ruhr-uni-bochum.de/Georg.Verweyen/pakete.html
%    Deaktiviert, da sonst "BPEL", "BPMN" usw. wirklich komisch aussehen.
%     Macht wohl nur bei geisteswissenschaftlichen Arbeiten Sinn.


% EN: amsmath teaks


% EN: Fixes bugs in AMS math
%     Corrently conflicts with unicode-math
% \usepackage{mathtools}

%\numberwithin{equation}{section}
%\renewcommand{\theequation}{\thesection.\Roman{equation}}

% EN: work-around ams-math problem with align and 9 -> 10. Does not work with glossaries, No visual changes.
%\addtolength\mathindent{1em}


% EN: For theorems, replacement for amsthm
\usepackage[amsmath,hyperref]{ntheorem}
\theorempreskipamount 2ex plus1ex minus0.5ex
\theorempostskipamount 2ex plus1ex minus0.5ex
\theoremstyle{break}
\newtheorem{definition}{Definition}[section]


% CTAN: https://ctan.org/pkg/lccaps
% Doc: http://texdoc.net/pkg/lccaps
%
% Required for DE/EN \initialism
\usepackage{lccaps}


% EN: Defintion of colors. Argument "hyperref" is not used as we do not want to change border colors of links: Links are not colored anymore.
% DE: Farbdefinitionen
\usepackage[dvipsnames]{xcolor}


% EN: Required for custom acronyms/glossaries style.
%     Left aligned Columns in tables with fixed width.
%     See http://tex.stackexchange.com/questions/91566/syntax-similar-to-centering-for-right-and-left
\usepackage{ragged2e}


% DE: Wichtig, ansonsten erscheint "No room for a new \write"
\usepackage{scrwfile}


% EN: Support for language-specific hyphenation
% DE: Neue deutsche Rechtschreibung und Literatur statt "Literature"
%     Die folgende Einstellung ist der Nachfolger von ngerman.sty
\ifdeutsch
  % DE: letzte Sprache ist default, Einbindung von "american" ermöglicht \begin{otherlanguage}{amercian}...\end{otherlanguage} oder \foreignlanguage{american}{Text in American}
  %     Siehe auch http://tex.stackexchange.com/a/50638/9075
  \usepackage[american,main=ngerman]{babel}
  % Ein "abstract" ist eine "Kurzfassung", keine "Zusammenfassung"
  \addto\captionsngerman{%
    \renewcommand\abstractname{Kurzfassung}%
  }
  \ifluatex
    % EN: conditionally disable ligatures. See https://github.com/latextemplates/scientific-thesis-template/issues/54
    %     for a discussion
    \usepackage[ngerman]{selnolig}
  \fi
\else
  % EN: Set English as language and allow to write hyphenated"=words
  %     `american`, `english` and `USenglish` are synonyms for babel package (according to https://tex.stackexchange.com/questions/12775/babel-english-american-usenglish).
  %      "english" has to go last to set it as default language
  \usepackage[ngerman,main=english]{babel}
  % EN: Hint by http://tex.stackexchange.com/a/321066/9075 -> enable "= as dashes
  \addto\extrasenglish{\languageshorthands{ngerman}\useshorthands{"}}
  \ifluatex
    % EN: conditionally disable ligatures. See https://github.com/latextemplates/scientific-thesis-template/issues/54
    %     for a discussion
    \usepackage[english]{selnolig}
  \fi
\fi
%


% EN: For easy quotations: \enquote{text}
%     This package is very smart when nesting is applied, otherwise textcmds (see below) provides a shorter command
%     Note that this package results in a warning when it is loaded before minted (actually fvextra).
% DE: Anführungszeichen
%     Zitate in \enquote{...} setzen, dann werden automatisch die richtigen Anführungszeichen verwendet.
%     Dieses package erzeugt eine Warnung, wenn es vor minted (genauer fvextra) geladen wird.
\usepackage{csquotes}


% EN: For even easier quotations: \qq{text}.
%     Is not smart in the case of nesting, but good enough for the most cases
\usepackage{textcmds}
\ifdeutsch
  % EN: German quotes are different. So do not use the English quotes, but the ones provided by the csquotes package.
  \renewcommand{\qq}[1]{\enquote{#1}}
\fi


% EN: extended enumarations
% DE: erweitertes Enumerate
\usepackage{paralist}


% DE: Gestaltung der Kopf- und Fußteilen

\usepackage[automark]{scrlayer-scrpage}

\automark[section]{chapter}
\setkomafont{pageheadfoot}{\normalfont\sffamily}
\setkomafont{pagenumber}{\normalfont\sffamily}

% DE: funktioniert nicht: Alle Linien sind hier weg
%\setheadsepline[.4pt]{.4pt}


% DE: Intelligentes Leerzeichen um hinter Abkürzungen die richtigen Abstände zu erhalten, auch leere.
%     Siehe commands.tex \gq{}
\usepackage{xspace}
% DE: Macht \xspace und \enquote kompatibel
\makeatletter
\xspaceaddexceptions{\grqq \grq \csq@qclose@i \} }
\makeatother


\newcommand{\eg}{e.\,g.,\ }
\newcommand{\ie}{i.\,e.,\ }


% EN: introduce \powerset - hint by http://matheplanet.com/matheplanet/nuke/html/viewtopic.php?topic=136492&post_id=997377
\DeclareFontFamily{U}{MnSymbolC}{}
\DeclareSymbolFont{MnSyC}{U}{MnSymbolC}{m}{n}
\DeclareFontShape{U}{MnSymbolC}{m}{n}{
  <-6>    MnSymbolC5
  <6-7>   MnSymbolC6
  <7-8>   MnSymbolC7
  <8-9>   MnSymbolC8
  <9-10>  MnSymbolC9
  <10-12> MnSymbolC10
  <12->   MnSymbolC12%
}{}
\DeclareMathSymbol{\powerset}{\mathord}{MnSyC}{180}


% EN: Package for the appendix
% DE: Anhang
\usepackage{appendix}
%[toc,page,title,header]
%


% EN: Graphics
% DE: Grafikeinbindungen
%
% EN: The parameter "pdftex" is not required
\usepackage{graphicx}
\graphicspath{{\getgraphicspath}}
\newcommand{\getgraphicspath}{graphics/}


% EN: Enables inclusion of SVG graphics - 1:1 approach
%    This is NOT the approach of https://ctan.org/pkg/svg-inkscape,
%     which allows text in SVG to be typeset using LaTeX
%     We just include the SVG as is.
\usepackage{epstopdf}
\epstopdfDeclareGraphicsRule{.svg}{pdf}{.pdf}{%
  inkscape -z -D --file=#1 --export-pdf=\OutputFile
}


% EN: Enables inclusion of SVG graphics - text-rendered-with-LaTeX-approach
%     This is the approach of https://ctan.org/pkg/svg-inkscape,
\newcommand{\executeiffilenewer}[3]{%
  \IfFileExists{#2}
  {
    %\message{file #2 exists}
    \ifnum\pdfstrcmp{\pdffilemoddate{#1}}%
      {\pdffilemoddate{#2}}>0%
      {\immediate\write18{#3}}
    \else
      {%\message{file up to date #2}
      }
    \fi%
  }{
    %\message{file #2 doesn't exist}
    %\message{argument: #3}
    %\immediate\write18{echo "test" > xoutput.txt}
    \immediate\write18{#3}
  }
}
\newcommand{\includesvg}[1]{%
  \executeiffilenewer{#1.svg}{#1.pdf}%
  {
    inkscape -z -D --file=\getgraphicspath#1.svg %
    --export-pdf=\getgraphicspath#1.pdf --export-latex}%
  \input{\getgraphicspath#1.pdf_tex}%
}


% EN: Enable typesetting values with SI units.
\ifdeutsch
  \usepackage[mode=text,group-minimum-digits=4]{siunitx}
  \sisetup{locale=DE}
\else
  \usepackage[mode=text,group-minimum-digits=4,group-separator={,}]{siunitx}
  \sisetup{locale=US}
\fi


% EN: Extensions for tables
% DE: Tabellenerweiterungen
\usepackage{array} %increases tex's buffer size and enables ``>'' in tablespecs
\usepackage{longtable}
\usepackage{dcolumn} %Aligning numbers by decimal points in table columns
\ifdeutsch
  \newcolumntype{d}[1]{D{.}{,}{#1}}
\else
  \newcolumntype{d}[1]{D{.}{.}{#1}}
\fi
\setlength{\extrarowheight}{1pt}


% DE: Eine Zelle, die sich über mehrere Zeilen erstreckt.
%     Siehe Beispieltabelle in Kapitel 2
\usepackage{multirow}


% DE: Fuer Tabellen mit Variablen Spaltenbreiten
%\usepackage{tabularx}
%\usepackage{tabulary}


% EN: Links behave as they should. Enables "\url{...}" for URL typesettings.
%     Allow URL breaks also at a hyphen, even though it might be confusing: Is the "-" part of the address or just a hyphen?
%     See https://tex.stackexchange.com/a/3034/9075.
% DE: Links verhalten sich so, wie sie sollen
%     Zeilenumbrüche bei URLs auch bei Bindestrichen erlauben, auch wenn es verwirrend sein könnte: Gehört der Bindestrich zur URL oder ist es ein Trennstrich?
%     Siehe https://tex.stackexchange.com/a/3034/9075.
\usepackage[hyphens]{url}
%
%  EN: When activated, use text font as url font, not the monospaced one.
%      For all options see https://tex.stackexchange.com/a/261435/9075.
% \urlstyle{same}
%
% EN: Hint by http://tex.stackexchange.com/a/10419/9075.
\makeatletter
\g@addto@macro{\UrlBreaks}{\UrlOrds}
\makeatother


% DE: Index über Begriffe, Abkürzungen
%\usepackage{makeidx} makeidx ist out -> http://xindy.sf.net verwenden


% DE: lustiger Hack fuer das Abkuerzungsverzeichnis
%     nach latex durchlauf folgendes ausfuehren
%     makeindex ausarbeitung.nlo -s nomencl.ist -o ausarbeitung.nls
%     danach nochmal latex
%\usepackage{nomencl}
%    \let\abk\nomenclature %Deutsche Ueberschrift setzen
%          \renewcommand{\nomname}{List of Abbreviations}
%        %Punkte zw. Abkuerzung und Erklaerung
%          \setlength{\nomlabelwidth}{.2\hsize}
%          \renewcommand{\nomlabel}[1]{#1 \dotfill}
%        %Zeilenabstaende verkleinern
%          \setlength{\nomitemsep}{-\parsep}
%    \makenomenclature


% EN: Logic for TeX - enables if-then-else in commands
% DE: Logik für TeX
%     FÜr if-then-else @ commands.tex
\usepackage{ifthen}


% EN: Code Listings
% DE: Listings
\usepackage{listings}
\lstset{language=XML,
  showstringspaces=false,
  extendedchars=true,
  basicstyle=\footnotesize\ttfamily,
  commentstyle=\slshape,
  % DE: Original: \rmfamily, damit werden die Strings im Quellcode hervorgehoben. Zusaetzlich evtl.: \scshape oder \rmfamily durch \ttfamily ersetzen. Dann sieht's aus, wie bei fancyvrb
  stringstyle=\ttfamily,
  breaklines=true,
  breakatwhitespace=true,
  % EN: alternative: fixed
  columns=flexible,
  numbers=left,
  numberstyle=\tiny,
  basewidth=.5em,
  xleftmargin=.5cm,
  % aboveskip=0mm, %DE: deaktivieren, falls man lstlistings direkt als floating object benutzt (\begin{lstlisting}[float,...])
  % belowskip=0mm, %DE: deaktivieren, falls man lstlistings direkt als floating object benutzt (\begin{lstlisting}[float,...])
  captionpos=b
}

\ifluatex
\else
  % EN: Enable UTF-8 support - see https://tex.stackexchange.com/q/419327/9075
  \usepackage{listingsutf8}
  \lstset{inputencoding=utf8/latin1}
\fi

\ifdeutsch
  \renewcommand{\lstlistlistingname}{Verzeichnis der Listings}
\fi


% EN: Alternative to listings could be fancyvrb. Can be used together.
% DE: Alternative zu Listings ist fancyvrb. Kann auch beides gleichzeitig benutzt werden.
\usepackage{fancyvrb}
%
% EN: Font size for the normal text
% DE: Groesse fuer den Fliesstext. Falls deaktiviert: \normalsize
%\fvset{fontsize=\small}
%
% DE: Somit kann im Text ganz einfach §verbatim§ text gesetzt werden.
%     Disabled, because UTF-8 does not work any more and lualatex causes issues
%\DefineShortVerb{\§}
%
% EN: Shrink font size of listings
\RecustomVerbatimEnvironment{Verbatim}{Verbatim}{fontsize=\footnotesize}
\RecustomVerbatimCommand{\VerbatimInput}{VerbatimInput}{fontsize=\footnotesize}
%
% EN: Hack for fancyvrb based on http://newsgroups.derkeiler.com/Archive/Comp/comp.text.tex/2008-12/msg00075.html
%     Change of the solution: \Vref somehow collidated with cleveref/varioref as the output of \Vref{} was "Abschnitt 4.3 auf Seite 85"; therefore changed to \myVref -- so completely removed
%     See https://tex.stackexchange.com/q/132420/9075 for more information.
\newcommand{\Vlabel}[1]{\label[line]{#1}\hypertarget{#1}{}}
\newcommand{\lref}[1]{\hyperlink{#1}{\FancyVerbLineautorefname~\ref*{#1}}}


% EN: Tunings of captions for floats, listings, ...
% DE: Bildunterschriften bei floats genauso formatieren wie bei Listings
%     Anpassung wird unten bei den newfloat-Deklarationen vorgenommen
%     https://www.ctan.org/pkg/caption2 is superseeded by this package.
\usepackage{caption}


% EN: Provides rotating figures, where the PDF page is also turned
% DE: Ermoeglicht es, Abbildungen um 90 Grad zu drehen
%     Alternatives Paket: rotating Allerdings wird hier nur das Bild gedreht, während bei lscape auch die PDF-Seite gedreht wird.
%     Das Paket lscape dreht die Seite auch nicht
\usepackage{pdflscape}


% EN: Required for proper environments of fancyvrb and lstlistings
%    There is also the newfloat pacakge (recommended by minted), but we currently have no expericene with that
% DE: Wird für fancyvrb und für lstlistings verwendet
\usepackage{float}
%
% EN: Alternative to float package
%\usepackage{floatrow}
% DE: zustäzlich für den Paramter [H] = Floats WIRKLICH da wo sie deklariert wurden paltzieren - ganz ohne Kompromisse
%     floatrow ist der Nachfolger von float
%     Allerdings macht floatrow in manchen Konstellationen Probleme. Deshalb ist das Paket deaktiviert.
%
% EN: See http://www.tex.ac.uk/cgi-bin/texfaq2html?label=floats
% DE: floats IMMER nach einer Referenzierung platzieren
%\usepackage{flafter}


% EN: Put footnotes below floats
%     Source: https://tex.stackexchange.com/a/32993/9075
\usepackage{stfloats}
\fnbelowfloat


% EN: For nested figures
% DE: Fuer Abbildungen innerhalb von Abbildungen
%     Ersetzt die Pakete subfigure und subfig - siehe https://tex.stackexchange.com/a/13778/9075
\usepackage[hypcap=true]{subcaption}


% EN: Extended support for footnotes
% DE: Fußnoten
%
%\usepackage{dblfnote}  %Zweispaltige Fußnoten
%
% Keine hochgestellten Ziffern in der Fußnote (KOMA-Script-spezifisch):
%\deffootnote[1.5em]{0pt}{1em}{\makebox[1.5em][l]{\bfseries\thefootnotemark}}
%
% Abstand zwischen Fußnoten vergrößern:
%\setlength{\footnotesep}{.85\baselineskip}
%
% EN: Following command disables the separting line of the footnote
% DE: Folgendes Kommando deaktiviert die Trennlinie zur Fußnote
%\renewcommand{\footnoterule}{}
%
\addtolength{\skip\footins}{\baselineskip} % Abstand Text <-> Fußnote
%
% Fußnoten immer ganz unten auf einer \raggedbottom-Seite
% fnpos kommt aus dem yafoot package
\usepackage{fnpos}
\makeFNbelow
\makeFNbottom


% EN: Variable page heights
% DE: Variable Seitenhöhen zulassen
\raggedbottom


% DE: Falls die Seitenzahl bei einer Referenz auf eine Abbildung nur dann angegeben werden soll,
%     falls sich die Abbildung nicht auf der selben Seite befindet...
\iftex4ht
  %tex4ht does not work well with vref, therefore we emulate vref behavior
  \newcommand{\vref}[1]{\ref{#1}}
\else
  \ifdeutsch
    \usepackage[ngerman]{varioref}
  \else
    \usepackage{varioref}
  \fi
\fi


% EN: More beautiful tables if one uses \toprule, \midrule, \bottomrule
% DE: Noch schoenere Tabellen als mit booktabs mit http://www.zvisionwelt.de/downloads.html
\usepackage{booktabs}
%
%\usepackage[section]{placeins}


% EN: Graphs and Automata
%
% TODO: Since version 3.0 (2013-10-01), it supports pdflatex via the auto-pst-pdf package
%       Requires -shell-escape
%\usepackage{gastex}


%\usepackage{multicol}

% DE: kollidiert mit diplomarbeit.sty
%\usepackage{setspace}


% DE: biblatex statt bibtex
\usepackage[
  backend       = biber, %biber does not work with 64x versions alternative: bibtex8
  %minalphanames only works with biber backend
  sortcites     = true,
  bibstyle      = alphabetic,
  citestyle     = alphabetic,
  giveninits    = true,
  useprefix     = false, %"von, van, etc." will be printed, too. See below.
  minnames      = 1,
  minalphanames = 3,
  maxalphanames = 4,
  maxbibnames   = 99,
  maxcitenames  = 2,
  natbib        = true,
  eprint        = true,
  url           = true,
  doi           = true,
  isbn          = true,
  backref       = true]{biblatex}

% enable more breaks at URLs. See https://tex.stackexchange.com/a/134281.
\setcounter{biburllcpenalty}{7000}
\setcounter{biburlucpenalty}{8000}

\bibliography{bibliography}
%\addbibresource[datatype=bibtex]{bibliography.bib}

%Do not put "vd" in the label, but put it at "\citeauthor"
%Source: http://tex.stackexchange.com/a/30277/9075
\makeatletter
\AtBeginDocument{\toggletrue{blx@useprefix}}
\AtBeginBibliography{\togglefalse{blx@useprefix}}
\makeatother

%Thin spaces between initials
%http://tex.stackexchange.com/a/11083/9075
\renewrobustcmd*{\bibinitdelim}{\,}

%Keep first and last name together in the bibliography
%http://tex.stackexchange.com/a/196192/9075
\renewcommand*\bibnamedelimc{\addnbspace}
\renewcommand*\bibnamedelimd{\addnbspace}

%Replace last "and" by comma in bibliography
%See http://tex.stackexchange.com/a/41532/9075
\AtBeginBibliography{%
  \renewcommand*{\finalnamedelim}{\addcomma\space}%
}

\DefineBibliographyStrings{ngerman}{
  backrefpage  = {zitiert auf S\adddot},
  backrefpages = {zitiert auf S\adddot},
  andothers    = {et\ \addabbrvspace al\adddot},
  %Tipp von http://www.mrunix.de/forums/showthread.php?64665-biblatex-Kann-%DCberschrift-vom-Inhaltsverzeichnis-nicht-%E4ndern&p=293656&viewfull=1#post293656
  bibliography = {Literaturverzeichnis}
}

% EN: enable hyperlinked author names when using \citeauthor
%     source: http://tex.stackexchange.com/a/75916/9075
\DeclareCiteCommand{\citeauthor}
{\boolfalse{citetracker}%
  \boolfalse{pagetracker}%
  \usebibmacro{prenote}}
{\ifciteindex
  {\indexnames{labelname}}
  {}%
  \printtext[bibhyperref]{\printnames{labelname}}}
{\multicitedelim}
{\usebibmacro{postnote}}

% EN: natbib compatibility
%\newcommand{\citep}[1]{\cite{#1}}
%\newcommand{\citet}[1]{\citeauthor{#1} \cite{#1}}
% EN: Beginning of sentence - analogous to cleveref - important for names such as "zur Muehlen"
%\newcommand{\Citep}[1]{\cite{#1}}
%\newcommand{\Citet}[1]{\Citeauthor{#1} \cite{#1}}

% DE: Blindtext. Paket "blindtext" ist fortgeschritterner als "lipsum" und kann auch Mathematik im Text (http://texblog.org/2011/02/26/generating-dummy-textblindtext-with-latex-for-testing/)
%     kantlipsum (https://www.ctan.org/tex-archive/macros/latex/contrib/kantlipsum) ist auch ganz nett, aber eben auch keine Mathematik
%     Wird verwendet, um etwas Text zu erzeugen, um eine volle Seite wegen Layout zu sehen.
\usepackage[math]{blindtext}


% EN: Make LaTeX logos available by commands. E.g., \lualatex
%     Disabled, because currently causes \not= already defined
%\usepackage{dtk-logos}

% quick replacement:
\newcommand{\LuaLaTeX}{Lua\LaTeX\xspace}
\newcommand{\lualatex}{\LuaLaTeX}

% DE: Neue Pakete bitte VOR hyperref einbinden. Insbesondere bei Verwendung des
%     Pakets "index" wichtig, da sonst die Referenzierung nicht funktioniert.
%     Für die Indizierung selbst ist unter http://xindy.sourceforge.net
%     ein gutes Tool zu erhalten.
%     Hier also neue packages einbinden.
% EN: Add new packages at this place.


% EN: Provides hyperlinks
%     Option "unicode" fixes umlauts in the PDF bookmarks - see https://tex.stackexchange.com/a/338770/9075
%
% DE: Erlaubt Hyperlinks im Dokument.
%     Alle Optionen nach \hypersetup verschoben, sonst crash
%     Siehe auch: "Praktisches LaTeX" - www.itp.uni-hannover.de/~kreutzm
\usepackage[unicode]{hyperref}


% EN: Define colors
% DE: Da es mit KOMA 3 und xcolor zu Problemen mit den global Options kommt MÜSSEN die Optionen so gesetzt werden.
%     Eigene Farbdefinitionen ohne die Namen des xcolor packages
\definecolor{darkblue}{rgb}{0,0,.5}
\definecolor{black}{rgb}{0,0,0}


% EN: Define color of links and more
\hypersetup{
  % have both title and number hyperlinking to content
  linktoc=all,
  bookmarksnumbered=true,
  bookmarksopen=true,
  bookmarksopenlevel=1,
  breaklinks=true,
  colorlinks=true,
  pdfstartview=Fit,
  pdfpagelayout=SinglePage, % DE: Alterntaive: TwoPageRight -- zweiseitige Darstellung: ungerade Seiten rechts im PDF-Viewer - siehe auch http://tex.stackexchange.com/a/21109/9075
  %pdfencoding=utf8, % EN: This is probably the same as passing the option "unicode" at \usepackage{hyperref}
  filecolor=darkblue,
  urlcolor=darkblue,
  linkcolor=black,
  citecolor=black
}


% EN: Abbreviations - has to be loaded after hyperref
% DE: Abkürzungsverzeichnis - muss nach hyperref geladen werden
%
% DE: siehe http://www.dickimaw-books.com/cgi-bin/faq.cgi?action=view&categorylabel=glossaries#glsnewwriteexceeded
\usepackage[acronym,indexonlyfirst,nomain]{glossaries}
\ifdeutsch
  \addto\captionsngerman % DE: siehe https://tex.stackexchange.com/a/154566
  {%
    \renewcommand*{\acronymname}{Abkürzungsverzeichnis}
  }
\else
  \renewcommand*{\acronymname}{List of Abbreviations}
\fi
\renewcommand*{\glsgroupskip}{}
%
% EN: Removed Glossarie as a table as a quick fix to get the template working again
%     See http://tex.stackexchange.com/questions/145579/how-to-print-acronyms-of-glossaries-into-a-table
%
\makenoidxglossaries


% EN: Extensions for references inside the document (\cref{fig:sample}, ...)
% DE: cleveref für cref statt autoref, da cleveref auch bei Definitionen funktioniert
\usepackage[capitalise,nameinlink,noabbrev]{cleveref}
\ifdeutsch
  \crefname{table}{Tabelle}{Tabellen}
  \Crefname{table}{Tabelle}{Tabellen}
  \crefname{figure}{\figurename}{\figurename}
  \Crefname{figure}{Abbildung}{Abbildungen}
  \crefname{equation}{Gleichung}{Gleichungen}
  \Crefname{equation}{Gleichung}{Gleichungen}
  \crefname{theorem}{Theorem}{Theoreme}
  \Crefname{theorem}{Theorem}{Theoreme}
  \crefname{listing}{\lstlistingname}{\lstlistingname}
  \Crefname{listing}{Listing}{Listings}
  \crefname{section}{Abschnitt}{Abschnitte}
  \Crefname{section}{Abschnitt}{Abschnitte}
  \crefname{paragraph}{Abschnitt}{Abschnitte}
  \Crefname{paragraph}{Abschnitt}{Abschnitte}
  \crefname{subparagraph}{Abschnitt}{Abschnitte}
  \Crefname{subparagraph}{Abschnitt}{Abschnitte}
\else
  \crefname{listing}{\lstlistingname}{\lstlistingname}
  \Crefname{listing}{Listing}{Listings}
\fi


% DE: Zur Darstellung von Algorithmen
%     Algorithm muss nach hyperref geladen werden
\usepackage[chapter]{algorithm}
\usepackage[]{algpseudocode}


% DE: Links auf Gleitumgebungen springen nicht zur Beschriftung,
%     Doc: http://mirror.ctan.org/tex-archive/macros/latex/contrib/oberdiek/hypcap.pdf
%     sondern zum Anfang der Gleitumgebung
\usepackage[all]{hypcap}


% DE: Deckblattstyle
%
\ifdeutsch
  \PassOptionsToPackage{language=german}{scientific-thesis-cover}
\else
  \PassOptionsToPackage{language=english}{scientific-thesis-cover}
\fi


% EN: Bugfixes packages
%\usepackage{fixltx2e} %Fuer neueste LaTeX-Installationen nicht mehr benoetigt - bereinigte einige Ungereimtheiten, die auf Grund von Rueckwaertskompatibilitaet beibahlten wurden.
%\usepackage{mparhack} %Fixt die Position von marginpars (die in DAs selten bis gar nicht gebraucht werden}
%\usepackage{ellipsis} %Fixt die Abstaende vor \ldots. Wird wohl auch nicht benoetigt.


% EN: Settings for captions of floats
% DE: Formatierung der Beschriftungen
%
\captionsetup{
  format=hang,
  labelfont=bf,
  justification=justified,
  %single line captions should be centered, multiline captions justified
  singlelinecheck=true
}


% EN: New float environments for listings and algorithms
%
% \floatstyle{ruled} % TODO: enabled or disabled causes no change - listings and algorithms are always ruled
%
\newfloat{Listing}{tbp}{code}[chapter]
\crefname{Listing}{Listing}{Listings}

\newfloat{Algorithmus}{tbp}{alg}[chapter]
\ifdeutsch
  \crefname{Algorithmus}{Algorithmus}{Algorithmus}
\else
  \crefname{Algorithmus}{Algorithm}{Algorithms}
  \floatname{Algorithmus}{Algorithm}
\fi



% EN: Various chapter styles
% DE: unterschiedliche Chapter-Styles
%     u.a. Paket fncychap

% Andere Kapitelueberschriften
% falls einem der Standard von KOMA nicht gefaellt...
% Falls man zurück zu KOMA moechte, dann muss jede der vier folgenden Moeglichkeiten deaktiviert sein.

%\usepackage[Sonny]{fncychap}

%\usepackage[Bjarne]{fncychap}

%\usepackage[Lenny]{fncychap}

%DE: Zur Aktivierung eines der folgenden Möglichkeiten ein Paar von "\iffalse" und "\fi" auskommentieren

\iffalse
  \usepackage[Bjarne]{fncychap}
  \ChNameVar{\Large\sf} \ChNumVar{\Huge} \ChTitleVar{\Large\sf}
  \ChRuleWidth{0.5pt} \ChNameUpperCase
\fi

\iffalse
  \usepackage[Rejne]{fncychap}
  \ChNameVar{\centering\Huge\rm\bfseries}
  \ChNumVar{\Huge}
  \ChTitleVar{\centering\Huge\rm}
  \ChNameUpperCase
  \ChTitleUpperCase
  \ChRuleWidth{1pt}
\fi

\iffalse
  \usepackage{fncychap}
  \ChNameUpperCase
  \ChTitleUpperCase
  \ChNameVar{\raggedright\normalsize} %\rm
  \ChNumVar{\bfseries\Large}
  \ChTitleVar{\raggedright\Huge}
  \ChRuleWidth{1pt}
\fi

\iffalse
  \usepackage[Bjornstrup]{fncychap}
  \ChNumVar{\fontsize{76}{80}\selectfont\sffamily\bfseries}
  \ChTitleVar{\raggedright\Large\sffamily\bfseries}
\fi

% EN: Complete different chapter style - self made

% Innen drin kann man dann noch zwischen
%   * serifenloser Schriftart (eingestellt)
%   * serifenhafter Schriftart (wenn kein zusaetzliches Kommando aktiviert ist) und
%   * Kapitälchen wählen
\iffalse
  \makeatletter
  %\def\thickhrulefill{\leavevmode \leaders \hrule height 1ex \hfill \kern \z@}

  %Fuer Kapitel mit Kapitelnummer
  \def\@makechapterhead#1{%
    \vspace*{10\p@}%
    {\parindent \z@ \raggedright \reset@font
      %Default-Schrift: Serifenhaft (gut fuer englische Dokumente)
      %A) Fuer serifenlose Schrift:
      \fontfamily{phv}\selectfont
      %B) Fuer Kapitaelchen:
      %\fontseries{m}\fontshape{sc}\selectfont
      %C) Fuer ganz "normale" Schrift:
      %\normalfont
      %
      \Large \@chapapp{} \thechapter
      \par\nobreak\vspace*{10\p@}%
      \interlinepenalty\@M
      {\Huge\bfseries\baselineskip3ex
        %Fuer Kapitaelchen folgende Zeile aktivieren:
        %\fontseries{m}\fontshape{sc}\selectfont
        #1\par\nobreak}
      \vspace*{10\p@}%
      \makebox[\textwidth]{\hrulefill}%    \hrulefill alone does not work
      \par\nobreak
      \vskip 40\p@
    }}

  %Fuer Kapitel ohne Kapitelnummer (z.B. Inhaltsverzeichnis)
  \def\@makeschapterhead#1{%
    \vspace*{10\p@}%
    {\parindent \z@ \raggedright \reset@font
      \normalfont \vphantom{\@chapapp{} \thechapter}
      \par\nobreak\vspace*{10\p@}%
      \interlinepenalty\@M
      {\Huge \bfseries %
        %Default-Schrift: Serifenhaft (gut fuer englische Dokumente)
        %A) Fuer serifenlose Schrift folgende Zeile aktivieren:
        \fontfamily{phv}\selectfont
        %B) Fuer Kapitaelchen folgende Zeile aktivieren:
        %\fontseries{m}\fontshape{sc}\selectfont
        #1\par\nobreak}
      \vspace*{10\p@}%
      \makebox[\textwidth]{\hrulefill}%    \hrulefill does not work
      \par\nobreak
      \vskip 40\p@
    }}
  %
  \makeatother
\fi


% DE: Minitoc-Einstellungen
%\dominitoc
%\renewcommand{\mtctitle}{Inhaltsverzeichnis dieses Kapitels}


% EN: Nicer paragraph line placement:
%     - Disable single lines at the start of a paragraph (Schusterjungen)
%     - Disable single lines at the end of a paragraph (Hurenkinder)
%     Normally, this is clubpenalty and widowpenalty, but using a package, it feels more non-hacky
\usepackage[all,defaultlines=3]{nowidow}
%
\displaywidowpenalty = 10000


% EN: Try to get rid of "overfull hbox" things and let text flow batter
%     See also
%       - http://groups.google.de/group/de.comp.text.tex/browse_thread/thread/f97da71d90442816/f5da290593fd647e?lnk=st&q=tolerance+emergencystretch&rnum=5&hl=de#f5da290593fd647e
%       - http://www.tex.ac.uk/cgi-bin/texfaq2html?label=overfull
\tolerance=2000
%
% EN: This could be increased to 20pt
\setlength{\emergencystretch}{3pt}
%
% EN: Suppress hbox warnings if less than 1pt
\setlength{\hfuzz}{1pt}


% EN: Fix names for algorithms in German
% DE: fuer algorithm.sty: - falls Deutsch und nicht Englisch.
\ifdeutsch
  \floatname{algorithm}{Algorithmus}
  \renewcommand{\listalgorithmname}{Verzeichnis der Algorithmen}
\fi




% Float-placements - http://dcwww.camd.dtu.dk/~schiotz/comp/LatexTips/LatexTips.html#figplacement
% and http://people.cs.uu.nl/piet/floats/node1.html
\renewcommand{\topfraction}{0.85}
\renewcommand{\bottomfraction}{0.95}
\renewcommand{\textfraction}{0.1}
\renewcommand{\floatpagefraction}{0.75}
%\setcounter{totalnumber}{5}

% EN: ensure that floats covering a whole page are placed at the top of the page
%    see http://tex.stackexchange.com/a/28565/9075
\makeatletter
\setlength{\@fptop}{0pt}
\setlength{\@fpbot}{0pt plus 1fil}
\makeatother



% DE: Bei Gleichungen nur dann die Nummer zeigen, wenn die Gleichung auch referenziert wird
%     Funktioniert mit MiKTeX Stand 2012-01-13 nicht. Deshalb ist dieser Schalter deaktiviert.
%
%\mathtoolsset{showonlyrefs}


% EN: Margins
% DE: Ränder
%     Viele Moeglichkeiten, die Raender im Dokument einzustellen.
%
%     Satzspiegel neu berechnen. Dokumentation dazu ist in "scrguide.pdf" von KOMA-Skript zu finden
%     Optionen werden bei \documentclass[] in ausarbeitung.tex mitgegeben.
% \typearea[current]{current} %neu berechnen, da neue Schrift eingebunden

%\usepackage{a4}
%\usepackage{a4wide}
%\areaset{170mm}{277mm} %a4:29,7hochx21mbreit

%Wer die Masse direkt eingeben moechte:
%Bei diesem Beispiel wird die Regel nicht beachtet, dass der innere Rand halb so gross wie der aussere Rand und der obere Rand halb so gross wie der untere Rand sein sollte
%\usepackage[inner=2.5cm, outer=2.5cm, includefoot, top=3cm, bottom=1.5cm]{geometry}

% EN: Package geometry to enlarge on page
%
%     Normally, geometry should not be used as the typearea package calculates the margins perfectly for printing
%     However, we want better screen-readable documents where the content does not "jump"
%     Thus, we fix the margins left and right to the same value
%
%     Source: http://www.howtotex.com/tips-tricks/change-margins-of-a-single-page/
%
\usepackage[
  left=3cm,right=3cm,top=2.5cm,bottom=2.5cm,
  headsep=18pt,
  footskip=30pt,
  includehead,
  includefoot
]{geometry}


% EN: Provides todo notes
% DE: schoene TODOs
\ifdeutsch
  \usepackage[colorinlistoftodos,ngerman]{todonotes}
\else
  \usepackage[colorinlistoftodos]{todonotes}
\fi
\setlength{\marginparwidth}{2,5cm}

\let\xtodo\todo
\renewcommand{\todo}[1]{\xtodo[inline,color=black!5]{#1}}
\newcommand{\utodo}[1]{\xtodo[inline,color=green!5]{#1}}
\newcommand{\itodo}[1]{\xtodo[inline]{#1}}


% EN: Enable footnotes in tables.
%     This package superseeds the 1997 package "footnote"
\usepackage{footnotehyper}
% TODO: The footnotehyper author recommends to enclose the respective area with \begin{savenotes} ... \end{savenotes}
\makesavenoteenv{tabular}
\makesavenoteenv{table}
% Reuse of footnotes, see http://tex.stackexchange.com/questions/10102/multiple-references-to-the-same-footnote-with-hyperref-support-is-there-a-bett
\crefformat{footnote}{#2\footnotemark[#1]#3}


% EN: pgfplots (optional if the ppackage is installed)
%     PGFPlots draws high-qual­ity func­tion plots in nor­mal or log­a­rith­mic scal­ing
\IfFileExists{pgfplots.sty}{
  \usepackage{pgfplots}
  % EN: highest version supported by overleaf as of 2018-03-16
  \pgfplotsset{compat=1.14}
}{}


% EN: pgfplotstable (optional if the ppackage is installed)
%     PGFPlots generates tables from csv files
\IfFileExists{pgfplotstable.sty}{
  \usepackage{pgfplotstable}
}{}


% EN: Package for creating graphics programmatically
\usepackage{tikz}


% EN: Package for creating uml diagramms
\usepackage{tikz-uml}


% EN: Forest: apgf/TikZ-based package for drawing linguistic trees - https://ctan.org/pkg/forest
\usepackage{forest}


% EN: Enable PlantUML listings in the environment "plantuml"
\IfFileExists{plantuml.sty}{
  \usepackage[output=latex]{plantuml}
}{}


% EN: Layout: bottoms of pages not aligned to each other
% DE: Der untere Rand darf "flattern"
\raggedbottom


% DE: Wie tief wird das Inhaltsverzeichnis aufgeschlüsselt
% 0 --\chapter
% 1 --\section % fuer kuerzeres Inhaltsverzeichnis verwenden - oder minitoc benutzen
% 2 --\subsection
% 3 --\subsubsection
% 4 --\paragraph
\setcounter{tocdepth}{1}


% EN: Fixes wrong spacing in the TOC.
%     Source: https://tex.stackexchange.com/a/33842/9075 -> comment by esdd
\RedeclareSectionCommand[tocnumwidth=2.8em]{section}


% DE: Angaben in die PDF-Infos uebernehmen
\makeatletter
\hypersetup{
  pdftitle={}, %Titel der Arbeit
  pdfauthor={}, %Author
  pdfkeywords={}, % CR-Klassifikation und ggf. weitere Stichworte
  pdfsubject={}
}
\makeatother


% EN: Higher compression of the output PDF
\pdfcompresslevel=9


% EN: Required for recent version of komascript, as some packges are not that compatible with KOMAScript as they should be
%     Has to be loaded at the *very* end, so we use "\AtEndPreamble" by etoolsbox
\usepackage{etoolbox}
\AtEndPreamble{\usepackage{scrhack}}


% EN: Provide tables over multiple pages
\usepackage{longtable}


% EN: Show LaTeX commands and their results in the document
%     Enables the command \PrintDemo
% See https://github.com/latextemplates/scientific-thesis-template/issues/82 for further discussion
\usepackage{latexdemo}


% DE: Fuer deutsche Texte: Weniger Silbentrennung, mehr Abstand zwischen den Woertern
\ifdeutsch
  \setlength{\emergencystretch}{3em} % Silbentrennung reduzieren durch mehr frei Raum zwischen den Worten
\fi



\usepackage[
  title={Implementing Variational Quantum Algorithms as Compositions of Reusable Microservice-based Plugins},
  author={Matthias Weilinger},
  type=master,
  institute=iaas, % or other institute names - or just a plain string using {Demo\\Demo...}
  course={Informatik},
  examiner={Prof.\ Dr.\ Dr.\ h.\ c.\ Frank Leymann},
  supervisor={M.Sc.\ Philipp Wundrack,\\M.Sc.\ Fabian Bühler},
  startdate={April 19, 2023},
  enddate={October 19, 2023}
]{scientific-thesis-cover}

% Hier stehen alle Abkürzungen
\newacronym{er}{ER}{error rate}
\newacronym{fr}{FR}{Fehlerrate}
\newacronym[plural={RDBMS},shortplural={RDBMS}]{rdbms}{RDBMS}{Relational Database Management System}
\newacronym[plural=VQAs, firstplural=Variational Quantum Algorithms (VQAs)]{vqa}{VQA}{Variational Quantum Algorithm}
\newacronym[plural=OFs, firstplural=Objective Functions (OFs)]{of}{OF}{Objective Function}


\makeindex

\begin{document}

%tex4ht-Konvertierung verschönern
\iftex4ht
  % tell tex4ht to create picures also for formulas starting with '$'
  % WARNING: a tex4ht run now takes forever!
  \Configure{$}{\PicMath}{\EndPicMath}{}
  %$ % <- syntax highlighting fix for emacs
  \Css{body {text-align:justify;}}

  %conversion of .pdf to .png
  \Configure{graphics*}
  {pdf}
  {\Needs{"convert \csname Gin@base\endcsname.pdf
      \csname Gin@base\endcsname.png"}%
    \Picture[pict]{\csname Gin@base\endcsname.png}%
  }
\fi

%\VerbatimFootnotes %verbatim text in Fußnoten erlauben. Geht normalerweise nicht.

% DE: wird fuer Tabellen benötigt (z.B. >{centering\RBS}p{2.5cm} erzeugt einen zentrierten 2,5cm breiten Absatz in einer Tabelle
\newcommand{\RBS}{\let\\=\tabularnewline}

% EN: To avoid issues with Springer's \mathplus
%     See also http://tex.stackexchange.com/q/212644/9075
\providecommand\mathplus{+}

% DE: typoraphisch richtige Abkürzungen
\newcommand{\zB}{z.\,B.\xspace}
\newcommand{\bzw}{bzw.\xspace}
\newcommand{\usw}{usw.\xspace}
\renewcommand{\dh}{d.\,h.\xspace}

% EN: from hmks makros.tex - \indexify
\newcommand{\toindex}[1]{\index{#1}#1}

% DE: Tipp aus "The Comprehensive LaTeX Symbol List"
\newcommand{\dotcup}{\ensuremath{\,\mathaccent\cdot\cup\,}}

% DE: Anstatt $|x|$ $\abs{x}$ verwenden.
%     Die Betragsstriche skalieren automatisch, falls "x" etwas größer sein sollte...
\newcommand{\abs}[1]{\left\lvert#1\right\rvert}

% DE: für Zitate
\newcommand{\citeS}[2]{\cite[S.~#1]{#2}}
\newcommand{\citeSf}[2]{\cite[S.~#1\,f.]{#2}}
\newcommand{\citeSff}[2]{\cite[S.~#1\,ff.]{#2}}
\newcommand{\vgl}{vgl.\ }
\newcommand{\Vgl}{Vgl.\ }

% EN: For the algorithmic package
\newcommand{\commentchar}{\ensuremath{/\mkern-4mu/}}
\algrenewcommand{\algorithmiccomment}[1]{\hfill $\commentchar$ #1}

% DE: Seitengrößen - Gegen Schusterjungen und Hurenkinder...
\newcommand{\largepage}{\enlargethispage{\baselineskip}}
\newcommand{\shortpage}{\enlargethispage{-\baselineskip}}

\newcommand{\initialism}[1]{%
  \ifdeutsch%
    \textsc{#1}\xspace%
  \else%
    \textlcc{#1}\xspace%
  \fi%
}
\newcommand{\OMG}{\initialism{OMG}}
\newcommand{\BPEL}{\initialism{BPEL}}
\newcommand{\BPMN}{\initialism{BPMN}}
\newcommand{\UML}{\initialism{UML}}

\pagenumbering{arabic}
\Titelblatt

%Eigener Seitenstil fuer die Kurzfassung und das Inhaltsverzeichnis
\deftriplepagestyle{preamble}{}{}{}{}{}{\pagemark}
%Doku zu deftriplepagestyle: scrguide.pdf
\pagestyle{preamble}
\renewcommand*{\chapterpagestyle}{preamble}



%Kurzfassung / abstract
%auch im Stil vom Inhaltsverzeichnis
\ifdeutsch
  \section*{Kurzfassung}
\else
  \section*{Abstract}
\fi

<Short summary of the thesis>

\cleardoublepage


% BEGIN: Verzeichnisse

\iftex4ht
\else
  \microtypesetup{protrusion=false}
\fi

%%%
% Literaturverzeichnis ins TOC mit aufnehmen, aber nur wenn nichts anderes mehr hilft!
% \addcontentsline{toc}{chapter}{Literaturverzeichnis}
%
% oder zB
%\addcontentsline{toc}{section}{Abkürzungsverzeichnis}
%
%%%

%Produce table of contents
%
%In case you have trouble with headings reaching into the page numbers, enable the following three lines.
%Hint by http://golatex.de/inhaltsverzeichnis-schreibt-ueber-rand-t3106.html
%
%\makeatletter
%\renewcommand{\@pnumwidth}{2em}
%\makeatother
%
\tableofcontents

% Bei einem ungünstigen Seitenumbruch im Inhaltsverzeichnis, kann dieser mit
% \addtocontents{toc}{\protect\newpage}
% an der passenden Stelle im Fließtext erzwungen werden.

\listoffigures
\listoftables

%Wird nur bei Verwendung von der lstlisting-Umgebung mit dem "caption"-Parameter benoetigt
%\lstlistoflistings
%ansonsten:
\ifdeutsch
  \listof{Listing}{Verzeichnis der Listings}
\else
  \listof{Listing}{List of Listings}
\fi

%mittels \newfloat wurde die Algorithmus-Gleitumgebung definiert.
%Mit folgendem Befehl werden alle floats dieses Typs ausgegeben
\ifdeutsch
  \listof{Algorithmus}{Verzeichnis der Algorithmen}
\else
  \listof{Algorithmus}{List of Algorithms}
\fi
%\listofalgorithms %Ist nur für Algorithmen, die mittels \begin{algorithm} umschlossen werden, nötig

% Abkürzungsverzeichnis
\printnoidxglossaries

\iftex4ht
\else
  %Optischen Randausgleich und Grauwertkorrektur wieder aktivieren
  \microtypesetup{protrusion=true}
\fi

% END: Verzeichnisse


% Headline and footline
\renewcommand*{\chapterpagestyle}{scrplain}
\pagestyle{scrheadings}
\pagestyle{scrheadings}
\ihead[]{}
\chead[]{}
\ohead[]{\headmark}
\cfoot[]{}
\ofoot[\usekomafont{pagenumber}\thepage]{\usekomafont{pagenumber}\thepage}
\ifoot[]{}


%% vv  scroll down for content  vv %%































%%%%%%%%%%%%%%%%%%%%%%%%%%%%%%%%%%%%%%%%%%%%%%%%%%%%%%%%%%%%%%%%%%%%%%%%%%%%%%
%
% Main content starts here
%
%%%%%%%%%%%%%%%%%%%%%%%%%%%%%%%%%%%%%%%%%%%%%%%%%%%%%%%%%%%%%%%%%%%%%%%%%%%%%%

\chapter{Notes what I have done so far}
\begin{itemize}
  \item Added recursive parsing of the plugin folders so that subfolders are also parsed
  \item created a callable plugin that gets the data parsed from its invoker via the database
  \item created a invoker that calls the callable plugin
  \item user can now select a plugin from the list of callable plugins, the list is narrowed down to the plugins that are compatible via the tag field
  \item creating a method to get the plugin name from the plugin URL
  \item made the callee plugin to multistep, to demonstrate that any amount of steps can be done in invoked plugin
  \item started with an optimizer plugin that gives the first frontend for the user to select the objective-function-plugin
  \item create an objective-function-plugin that takes means squared error as an objective function
  \item todo: creating a method to get the plugin metadata from the plugin URL (this is needed in order to get the entry points of the plugin)
  \item identified three key problems:
  \begin{itemize}
    \item Reliable way to pass a callback function to the callee plugin
    \item A way to get a list of interaction endpoints of the callee plugin
    \item A way to get the plugin metadata from the plugin URL
  \end{itemize}
  \item 19.05: I am currently working on a big problem
  \begin{itemize}
    \item I have the optimizer plugin which should call the objective-function-plugin so that it can ask the user for the hyperparameters of the objective function
    \item For that reason I need to add a next step as a celery task.
    \item The thing is i don't want to handle the next task like other multistep plugins where they share a db id since the objective-function-plugin should be able to stand on its own.
    \item Therefore, I pass a callback function to the objective-function-plugin which it should call when it is done with the setup.
    \item works all fine like that
    \item The problem right now is how i call the objective-function-plugin from the optimizer plugin
    \item I need to add it as a step, which is usually done via the add\_step celery task
    \item This task though needs a db\_id which I don't want to add.
    \item When adding none it works to call the objective-function-plugin and its also possbile to call the callback function
    \item but the problem is that with the add\_step task the celery task is not called asyncronously
    \item therefore when the callback function is called the optimizer plugin is not yet finished with the add\_step task
    \item usually one would go in the called multistep and do a clear\_previous\_steps call to make sure that the previous steps are finished
    \item but this call does not work since the objective-function-plugin does not have a db\_id
    \item thinking right now....
    \item maybe we can finish the task in the callback function of the optimizer plugin?
    \item Still 19.05 here we are again
    \item I have now a working solution for the problem
    \item the proposed solution of clearing the previous steps in the callback function did work
    \item the problem was that i forgot to add the call db\_task.save(cammit=True) in the callback function
    \item this is needed to save the changes to the database without it the cleared variable is not saved
  \end{itemize}
  \item 20.05: Today we give the objective-function-plugin the ability to calculate the loss function
  \begin{itemize}
    \item On callback we give the optimizer plugin the url of the calculateLoss function url
    \item The optimizer plugin then calls the calculateLoss function
  \end{itemize}
  \item 22.05: we continue with the upper
  \begin{itemize}
    \item it is now possible to calculate the loss function
    \item we simply call the calculation enpoint with a post request
    \item we pass all the necessary data in the body of the request
    \item we have a special schema for that
    \item current problem: passing the data to the minimize function of scipy.optimize
    \item we have the hyperparameters as a dict to keep it as generic as possible
    \item but the minimize function needs the hyperparameters as a list
    \item and also the loss function needs the hyperparameters as a dict, so lets see how we can solve this
    \item we maybe did it by just passing the hyperparameters as a dict to both of the functions
    \item now we have the problem that the content type of the input file is not correctly set by postman
    \item we just change the code to accept the content type but lets not forget to change it back
  \end{itemize}
  \item 23.05: today we do some cleanup
  \begin{itemize}
    \item make the import relative so that it works with the docker container
    \item remove all no ops tasks
  \end{itemize}
  \item 02.06:
  \begin{itemize}
    \item we want to skip the optimizer UI since we do not need any more input data
    \item tried to set the cleared value of the created step to true, but it did not work, since only the process step is started is by clicking submit in the ui
    \item trying to chain the tasks directly in the callback function, which works
  \end{itemize}
  \item 05.06:
  \begin{itemize}
    \item now I want to call the objective-function-plugin via the entry points that I get through the metadata, this is needed to make the plugins more generic
    \item I have a problem where the shared schemas can not be imported since the NumPy package cannot be found, added the NumPy package to the requirements for the opt plugin, but this did not solve the problems
    \item maybe this will be solved by creating a coordinator plugin that lives in the top level of the plugin folder
  \end{itemize}
  \item 12.06:
  \begin{itemize}
    \item Creating a top level coordinator plugin was not the solution. If an init file is present no further plugins will be loaded from the folder
    \item the shared schemas are now part of the coordinator plugin
    \item Created a new infrastructure for the plugins with a coordinator plugin (diagram will be needed)
    \item today we solve the following problem:
    \item coordinator waits for the optimizer plugin to finish and writes the result to a file
    \item for this the coordinator polls the task api to check if the minimizer task is finished
    \item when it is it writes the result to a file
    \item I should move the callback URL away from the query parameters to the body of the request, maybe to the form
    \item next I should read about how neural networks work and how to minimize them
  \end{itemize}
  \item 13.06:
  \begin{itemize}
    \item i have made a decision on how to handle the callback url from the ui to the processing endpoint
    \item until now the callback url was passed as a query parameter
    \item now i want to pass it into the form that is rendered by the UI
    \item it should be a hidden field
    \item i have to make and change so that i can pass multiple schemas to the render function and set which fields should be hidden
  \end{itemize}
  \item 16.06:
  \begin{itemize}
    \item when an invoked plugin now makes a callback to its invoker it only passes back the endpoint for the calculation endpoint.
    \item it does not pass any hyperparameters back since it should own the hyperparameters and not the invoker
    \item i now want to have a look of how loss functions are called in python and how hyperparameters are passed to them
    \item with this information I want to create interaction endpoints for the objective-function-plugin that are completely generic, i.e. the hyperparameters are passed to the endpoint as a dict
    \item as an example of how a generic method could look like i will have a look at the scipy.optimize.minimize function
    \item I now interaction endpoints as and additional list to entry points
  \end{itemize}
  \item 25.06:
  \begin{itemize}
    \item i have added a functionality where I use the blinker library to create a signal that is emitted when the status of a task changes
    \item this is used by the coordinator plugin to check if the minimization task is finished
    \item the coordinator plugin passes the callback URL to the minimization calculation endpoint
    \item this endpoint registers the URL to the db
    \item the signal handlers makes a post request to this URL when the status of the task changes
    \item I have moved all shared schemas into a separate folder
    \item I have moved all utilities concerning plugin interactions into a separate folder
  \end{itemize}
  \item 26.06:
  \begin{itemize}
    \item I developed a way to pass any amount of parameters to the processing endpoint of the invoked plugin UI
    \item I added an interaction endpoint that is used to invoke the invoked plugin
    \item this interaction endpoint allows any number of parameters
    \item the endpoint saves these parameters to the database
    \item it then adds the entry points of the invoked plugin as a step
    \item it passes the processing URL with the db id as a query parameter to the invoked plugin
    \item the invoked plugin then calls the processing endpoint with the db id as a query parameter
    \item here it starts a new db task with the arguments that were passed to the invoked plugin
  \end{itemize}
\end{itemize}


\chapter{Define the plugins}
\begin{itemize}
  \item \textbf{ObjectiveFunction}: This plugin should have the following steps
  \begin{itemize}
    \item \textbf{ /get hyperparameterUI }: This step should let the user select the hyperparameters of the objective function
    \item \textbf{ /post ObjectiveFunctionSetup}: This step should setup the objective function with set a database id for future reference of the parameter.
    Then it should store the following information to the database:
    \begin{itemize}
      \item hyperparameters
      \item more stuff?? %%% TODO specify what more stuff
    \end{itemize}
    Then it should call the optimizer callback function to indicate that the setup is done. Pass the url of the calculateLoss function as a parameter.
    \item \textbf{ /post CalculateLoss (dbID) }: this step should trigger the calculation of the loss function and should return it.
  \end{itemize}
  \item \textbf{Optimizer}: This plugin should have the following steps:
    \begin{itemize}
      \item \textbf{ /get setup UI }: Let the user select the objective-function-plugin, dataset, target variable, and the optimization algorithm
      \item \textbf{ /post setup }: This step should setup the optimizer with set a database id for future reference of the parameter.
      Then it should call the objective function entry point to setup the objective function. Pass the url of the optimizer callback function as a parameter.
      \item \textbf{ /post callback }: This endpoint should be called by the objective-function-plugin to indicate that the setup is done.
      It should then start the optimization process.
      \item \textbf{ /post optimize }: This step should trigger the optimization process.
      It should loop the optimization function until a stop condition is met.
      In each iteration it should call the objective-function-plugin to calculate the loss function.
    \end{itemize}
\end{itemize}

\chapter{Introduction}

This thesis starts with \cref{chap:k2}.

We can also typeset \verb|<text>verbatim text</text>|.
Backticks are also rendered correctly: \verb|`words in backticks`|.

\chapter{Background}
\label{chap:background}

Before diving deep into the intricacies of QHana and its innovative plugin interactions, it's beneficial to have a solid grasp on some foundational concepts.
This chapter serves as a brief primer, covering the essentials of optimization algorithms, quantum computing, \glspl{vqa}, REST and QHAna.
By understanding these basics, the subsequent sections become more accessible and meaningful, since this thesis combines these concepts in novel ways.
Whether familiar with these topics or encountering them for the first time, this chapter ensures a smooth transition into the core content of the thesis.

\section{Optimization Algorithms}
\label{sec:optimizationAlgorithms}
Optimization is a powerful tool that's ubiquitous in various scientific and technological domains.
At its core, optimization is about finding the best solution from a set of possible choices.
This section provides a snapshot of optimization's fundamental principles, paving the way for its deeper exploration in the context of \glspl{vqa} and finally microservice based \glspl{vqa}.

\subsection{Objective Functions}
\label{subsec:objectiveFunctions}
\glspl{of}, serving as the cornerstone of optimization problems, form the foundation for a wide range of computational algorithms and models.
They provide a metric to gauge the performance of a given model, solution, or set of parameters.
\glspl{of} are the heart of many optimization problems.
The goal is to minimize or maximize these functions depending on the context and requirements \cite{Weinan2017}.
In the context of \glspl{vqa}, the primary objective is to minimize the function.

The core inputs to a \gls{of} typically encompass data points (denoted as $x$), corresponding labels or outcomes (represented by $y$), and a set of parameters or weights (often symbolized by $\theta$ or $w$).
These parameters dictate how the model responds to the input data and makes its predictions.
Additionally, certain \glspl{of} may also include hyperparameters as input, which control the behavior and complexity of the model.
In the context of optimization problems, the role of an \gls{of} is to capture both the problem we're attempting to solve and the strategy by which we're trying to solve it.
It provides a measure of the 'goodness' or 'fitness' of our current solution or parameters, and the aim is to adjust these parameters to improve this measure.

One example of an \gls{of} is the Lasso (Least Absolute Shrinkage and Selection Operator) Loss function.
The Lasso loss function is represented as:
\[
L(Y, X, W, \lambda) = ||Y - XW||^2_2 + \lambda ||W||_1
\]
In this equation:

\begin{itemize}
  \item \(Y\) is the vector of observed values.
  \item \(X\) is the matrix of input data points.
  \item \(W\) is the vector of weights, the parameters of the model.
  \item \(\lambda\) is the regularization parameter, a non-negative hyperparameter.
\end{itemize}

This function consists of two terms:
\begin{enumerate}
  \item The first term \(||Y - XW||^2_2\) is the mean squared error between the predicted and actual outcomes.
  It measures the discrepancy between the model's predictions and the true values.
  \item The second term \(\lambda ||W||_1\) is a regularization term, where \(||W||_1\) represents the L1 norm (sum of absolute values) of the weight vector.
  This term penalizes the absolute size of the coefficients, encouraging them to be small.
\end{enumerate}
The hyperparameter \(\lambda\) governs the trade-off between these two terms.
When \(\lambda = 0\), the \gls{of} reduces to ordinary least squares regression, and the weights are chosen to minimize the mean squared error alone.
As \(\lambda\) increases, more weight is given to the regularization term, and the solution becomes more sparse (i.e., more of the weights are driven to zero).
This can help to prevent overfitting by effectively reducing the complexity of the model \cite{ShalevShwartz2014}

\subsection{Minimization Functions}
\label{subsec:minimizationFunctions}
Minimization functions, often referred to as optimization algorithms, play a pivotal role in a vast array of computational models and algorithms.
In essence, they serve to iteratively enhance the parameters of a model to reduce the value of the \gls{of}.
The goal of these minimization functions is to find the optimal set of parameters that yield the lowest possible value of the \gls{of} within the constraints of the problem \cite{Nocedal2006}.

The process of optimization involves a search through the parameter space.
This search can be visualized as navigating a landscape of hills and valleys, with each point in the landscape corresponding to a different set of parameters, and the height at each point representing the value of the \gls{of}.
The goal of the minimization function is to find the lowest point in this landscape, corresponding to the minimum value of the \gls{of} \cite{Goodfellow2017}.

The core inputs to a minimization function are the initial parameters of the model or weights (denoted as \(\theta\) or \(w\)),
the \gls{of} that needs to be minimized, and the gradients of the \gls{of} with respect to the parameters.
Additionally, certain minimization functions may also include hyperparameters as input, which control the behavior and complexity of the optimization process \cite{Virtanen2020}.
For instance, the learning rate is a typical hyperparameter that determines the step size in each iteration of the optimization process.

There are numerous minimization functions used in computational problems, each with its own strengths and weaknesses.
These range from simple methods such as gradient descent, to more complex ones such as the Newton's method, stochastic gradient descent (SGD), RMSprop, and Adam.

One of the most fundamental and widely used minimization functions is the Gradient Descent.
To find a local minimum of a function using gradient descent, one takes steps proportional to the negative of the gradient (or approximate gradient) of the function at the current point.

The update rule of gradient descent is given as:

\[
\theta_{t+1} = \theta_t - \alpha \nabla F(\theta_t)
\]

In this formula:

\begin{itemize}
  \item \(\theta_{t+1}\) represents the parameters at the next time step.
  \item \(\theta_t\) represents the current parameters.
  \item \(\alpha\) is the learning rate, a positive scalar determining the size of the step.
  \item \(\nabla F(\theta_t)\) is the gradient of the of.
\end{itemize}

Here, the \gls{of} \(F\) is assumed to be a differentiable function.
The gradient \(\nabla F(\theta_t)\) provides the direction of the steepest ascent at the point \(\theta_t\), and \(-\nabla F(\theta_t)\) provides the direction of steepest descent.
By taking a step in this direction, we move towards the minimum of the function.

The size of the steps taken is determined by the learning rate \(\alpha\), which is a hyperparameter that must be set before the learning process begins.
The learning rate controls how fast or slow we move towards the optimal weights.
If the learning rate is very large, we may skip the optimal solution.
If it is too small, we may need too many iterations to converge to the best values.

The choice of minimization function can significantly influence the efficiency and success of the optimization process.
While some minimization functions may perform well on certain problems, they may not yield similar results on others.
Therefore, understanding the underlying mechanisms of these functions and their suitability to the specific problem at hand is crucial.

\section{Quantum Computing}
\label{sec:quantumComputing}
Quantum computing is a cutting-edge field that exploits the principles of quantum mechanics to process information.
Unlike classical computers that use bits (0s and 1s) to store and process information, quantum computers use quantum bits, or "qubits."
Qubits, through the phenomena of superposition and entanglement, can exist in multiple states at once and be correlated with each other in ways that classical bits cannot \cite{Nielsen2010}.

Superposition allows a qubit to be in a state that is a combination of both 0 and 1, with a certain probability for each.
This property enables quantum computers to perform many calculations simultaneously, vastly increasing their potential computational power.
Entanglement, on the other hand, allows qubits that are entangled to be intimately linked regardless of the distance separating them.
A change in the state of one will instantaneously affect the state of the other, a phenomenon that Einstein famously called "spooky action at a distance" \cite{Einstein1935}.
This property is essential for many quantum algorithms, quantum error correction, and quantum teleportation, making it a fundamental resource in quantum information processing \cite{Nielsen2010,Preskill1998}.


\section{Variational Quantum Algorithms}
\label{sec:variationalQuantumAlgorithms}

\glspl{vqa} bring together the principles of quantum computing and optimization in a unique and powerful way.
They are a class of hybrid quantum-classical algorithms that leverage the strengths of both quantum and classical computing to solve complex problems \cite{McClean2016}.

The main concept of \glspl{vqa} is to use a sequence of quantum operations (a "quantum circuit") controlled by certain parameters.
These parameters are adjusted using classical optimization techniques with the aim of solving a specific problem.
This problem, in many cases, involves finding the lowest energy state, or "ground state", of a quantum system, a problem that maps to finding the minimum of a particular function \cite{Peruzzo2013}.

By leveraging classical optimization algorithms, \glspl{vqa} become more resistant to quantum errors, as the majority of the computation is performed on a classical computer.
This combination of quantum and classical resources makes \glspl{vqa} a promising type of algorithm for near-term quantum devices \cite{Moll2017}.

\section{QHana}
\label{sec:qhana}

QHana, short for Quantum Humanities Analysis tool, is a unique application developed in the domain of Digital Humanities (DH).
It provides a platform for users to experiment with various machine learning algorithms on specified datasets.
With the advent of quantum computers and their availability in the cloud, QHana has become a viable tool for evaluating the potential advantages of quantum algorithms in the DH community.

QHana is designed to be extensible, allowing the integration of new data sources and quantum algorithms as plugins.
However, plugins are specifically built for specific applications, limiting their reusability in other applications.
Moreover, plugins for an application have to be developed in the same programming language as the application.
Even if a plugin can be reused in another application, its UI has to adapt to the new application, otherwise users may fail to understand the plugin's functionality.
To address this limitation, QHana is built on a novel concept of Reusable Microservice-based Plugins (RAMPs).
This allows microservices with an integrated UI to be used as plugins by multiple applications, enhancing the reusability of the plugins across different applications \cite{Buehler2022}.

In the context of \glspl{vqa}, this thesis aims to extend the capabilities of QHana by developing a novel approach where plugins can interact with each other.
By leveraging the principles of quantum computing and optimization, \glspl{vqa} can be implemented as interchangeable plugins within the QHana platform.
This means that different components of a \gls{vqa}, such as the minimizer and the \gls{of}, can be developed as separate plugins.
These plugins can then interact, communicate, and collaborate to achieve the desired optimization results.
This modular approach not only enhances the flexibility and scalability of the system but also promotes reusability, as individual plugins can be utilized in various combinations to tackle different optimization problems.


\section{RESTful API Design}
In the evolving landscape of software design, microservices have emerged as a preferred architectural style, prized for their modularity, scalability, and independent deployability.
At the heart of QHAna's design is a microservices-based plugin approach, which facilitates its dynamic and extensible nature.
Central to the orchestration of these microservices is the application of RESTful API design.
Representational State Transfer (REST) is an architectural style that sets forth constraints for creating web services.
RESTful APIs, built upon these constraints, are pivotal in ensuring seamless communication between individual microservices, thereby enabling efficient data exchange and service integration.

In platforms like QHAna, where the next step is on plugin interactions, the role of RESTful APIs becomes even more pronounced.
Given that each plugin in QHAna is essentially a microservice, the ability for these plugins to interact, share data, and even leverage functionalities from one another is crucial.
A well-designed RESTful interface ensures that these plugins can dynamically discover and communicate with each other, making the system adaptable and extensible.
This not only promotes interoperability but also enhances the overall resilience, robustness and security of the platform.

For those seeking a deeper dive into the intricacies of RESTful design in microservices architectures, influential works by Fielding \cite{Fielding2000} and practical insights by Richardson and Amundsen \cite{Richardson2013} are recommended.

\chapter{Problem Statement and Objectives}
\label{chap:problem}

Optimization algorithms, with their ability to find the best possible solution from a set of feasible solutions, play a pivotal role in numerous computational domains.
Variational Quantum Algorithms (VQAs) are a subset of these algorithms that leverage quantum computing principles, particularly in the realm of objective functions and minimization techniques.
However, the true potential of optimization, and by extension VQAs, is often hindered by rigid platforms where the components of these algorithms are tightly integrated, limiting adaptability and innovation.

QHana, with its unique environment tailored for experimenting with a myriad of machine learning and quantum algorithms, presents an opportunity to redefine this paradigm.
Yet, its current architecture does not fully exploit the modular benefits that can be achieved by decoupling the components of optimization algorithms.
Furthermore, while developing this modular framework, it's essential to allow for plugins to interact with each other.
This interaction-centric concept, once established, can be universally applied across QHana, not just for optimization but for any scenario where plugin interaction is required.

\textbf{Problem Statement:}
How can we design and implement a modular framework within QHana that allows components of optimization algorithms, specifically objective functions and minimization functions, to be encapsulated as distinct, interchangeable plugins?
Furthermore, how can these plugins, especially in the context of VQAs, be structured to communicate and collaborate seamlessly?

This problem encompasses several challenges:

\begin{itemize}
    \item \textbf{Communication:} Establishing a robust communication mechanism that enables interaction, data sharing, and collaboration among these plugins.
    \item \textbf{Interchangeability:} Designing a system where different objective function and minimization plugins can be effortlessly swapped, ensuring adaptability in optimization and VQAs.
    \item \textbf{Standardization:} Implementing a consistent interface for these plugins, ensuring uniformity and compatibility across various objective function and minimization plugins.
    \item \textbf{User Experience:} Providing an intuitive environment where users can easily select, interchange, and experiment with different optimization components tailored to their needs.
\end{itemize}

Addressing this problem is essential to enhance the capabilities of QHana, transforming it into a dynamic, adaptable, and user-centric platform for optimization and VQAs.
The subsequent sections of this thesis will delve into the methodologies, implementations, and evaluations related to this problem.

\chapter{Related Work}

\cite{Beisel2023} \cite{Beisel2023a} \cite{Thullier2021}

\chapter{Methodology}
\label{chap:methodology}
\section{Concept Development for Plugin Interaction}
\label{sec:conceptDevelopment}

\chapter{Implementation}
\label{chap:implementation}
\section{Directory Structure and Plugin Loading}
\label{sec:directoryStructure}

QHana maintains two primary directories for plugin management: the \textit{plugins} and the \textit{stable plugins} directories.
The former hosts plugins undergoing development, while the latter contains plugins that are deployment-ready.
Within these directories, individual plugins are neatly organized into dedicated subfolders.
Upon initiation, QHana scans and loads plugins from these designated subfolders.

The focus of this thesis, the optimization plugin, resides in the \textit{plugins/optimization} directory.
To promote clarity and a structured layout, further divisions were made:

\begin{itemize}
  \item \textit{plugins/optimization/coordinator} – For the main optimization plugin.
  \item \textit{plugins/optimization/objective\_functions} – Where each \gls{of} plugin occupies its respective subfolder.
  \item \textit{plugins/optimization/minimizer} – Where each minimizer plugin occupies its respective subfolder.
\end{itemize}

Given that QHana's native architecture doesn't support the direct loading of plugins from nested subdirectories, a recursive plugin loader was developed for this purpose.
This loader traverses through the \textit{plugins} directory and its subdirectories.
The presence of an \textit{\_\_init\_\_.py} file within a folder acts as a confirmation for the plugin's legitimacy.
If any plugin needs to be excluded from the loading process, a \textit{.ignore} file is placed in its respective folder to act as a loading deterrent.

To ensure that the recursive process doesn’t compromise system performance, the recursion depth is capped at 4.
This threshold sufficiently accommodates the current plugin structure but can be adjusted upwards if future needs arise.

The folder hierarchy is further enriched by the inclusion of an \textit{interaction\_utils} directory, dedicated to housing utility functions for generalized plugin interactions.
Additionally, there's a \textit{shared} directory, which stores data structures and schemas utilized across the plugins related to optimization plugin.
A visual representation of the final folder structure, with all plugins implemented for this thesis, is shown in \cref{fig:folderStructure}.


\begin{figure}[h!]
  \dirtree{%
      .1 plugins.
      .2 optimizer.
      .3 coordinator.
      .3 interaction\_utils.
      .3 minimizer.
      .4 scipy\_minimizer.
      .4 scipy\_minimizer\_grad.
      .3 objective\_functions.
      .4 hinge\_loss.
      .4 neural\_network.
      .4 ridge\_loss.
      .3 shared.
  }
  \caption{QHana Plugin Folder Structure.}
  \label{fig:folderStructure}
\end{figure}


\section{Plugin Infrastructure}
\label{sec:pluginInfrastructure}
\section{Plugin Interaction}
\label{sec:pluginInteraction}
\section{Minimizer Plugin}
\label{sec:minimizerPlugin}
\section{Objective Function Plugins}
\label{sec:objectiveFunctionPlugins}


\chapter{Evaluation}
\label{chap:evaluation}
\section{Sample Machine Learning Experiment}
\label{sec:sampleMachineLearningExperiment}
\section{Interchangeability of Plugins}
\label{sec:interchangeabilityOfPlugins}
\section{Performance Analysis}
\label{sec:performanceAnalysis}

The performance of the system is evaluated by measuring the time the system takes to complete a machine learning experiment.
The time that it takes for a user to input the data is not measured.
This benchmark should only measure the time that it takes for the system to complete the experiment.
We compare the performance of the plugin based system with the performance of a jupyter notebook based system.
There are several steps that are benchmarked in the plugin based system, since there is user interaction between the steps.
\begin{itemize}
  \item The time that it takes after the user has selected the plugins until the user can input the hyperparameters for the of plugin.
  \item The time that it takes after the user has input the of plugin hyperparameters until the user can input the hyperparameters for the minimizer plugin.
  \item The time that it takes after the user has input the minimizer plugin hyperparameters until the user gets the result of the experiment.
\end{itemize}
The times of those steps are summed up to get the total time that it takes for the plugin based system to complete the experiment.
The time that it takes for the jupyter notebook based system is measured from the start of the notebook until the user gets the result of the experiment.
The time is measured by adding a timestamp at the start of the step and at the end of the step.
The difference between the two timestamps is the time that it takes for the step to complete.
The time that it takes for the plugin based system to complete the experiment is compared to the time that it takes for the jupyter notebook based system to complete the experiment.
In order to get a more accurate result, the experiment is repeated several times and the average time is taken.
To make it as comparable as possible, the same datasets and hyperparameters are used for both systems.
The output file of the experiment should also have the same exact format.

Since the benchmark of the plugin based system shows worse performance than the jupyter notebook based system, we want to explore where the most time is spent.
Therefore, we measure the time it takes to call to calculate the loss function.
In case of the plugin based system this includes the network latency between the plugins.
We want to see how much time is spent in the network and how much time is spent in the actual calculation of the loss function.

For more sophisticated loss functions (in this case the neural network) we want to see how much time is spent for the setup of the neural network.
Since a microservice does not have state, the neural network has to be setup for every call of to calculate loss function.

\section{Hardware and System Specifications}

The benchmarks were conducted on a MacBook Pro. The detailed specifications of the machine are as follows:

\begin{itemize}
    \item \textbf{Model:} MacBookPro18,1
    \item \textbf{Processor (CPU):} Apple M1 Pro
    \item \textbf{Memory (RAM):} 32 GB (hw.memsize: 34359738368 bytes)
    \item \textbf{Graphics Processing Unit (GPU):}
    \begin{itemize}
        \item Chipset Model: Apple M1 Pro
        \item Type: GPU
        \item Bus: Built-In
        \item Total Number of Cores: 16
        \item Metal Support: Metal 3
    \end{itemize}
    \item \textbf{Operating System:} macOS, Version 13.4.1, Build Version: 22F770820d
\end{itemize}


\section{Accuracy Analysis}
\label{sec:accuracyAnalysis}
In terms of accuracy we compare the results of the plugin based system with the results of the jupyter notebook based system.
The results of the two systems should be the same, since the same datasets and hyperparameters are used.
Also, we should see the same results since the same actual code is executed in both systems, only the way of calling the code is different.
The results are compared by calculating the mean squared error between the two results.



\chapter{Discussion}
\label{chap:discussion}
\section{Achievements and Contribution}
\label{sec:achievementsAndContribution}
\section{Limitations}
\label{sec:limitations}


\chapter{Conclusion and Outlook}
\label{chap:zusfas}

\section{Outlook}

\printbibliography

All links were last followed on October 15, 2023.

\appendix

\pagestyle{empty}
\renewcommand*{\chapterpagestyle}{empty}
\Versicherung
\end{document}
